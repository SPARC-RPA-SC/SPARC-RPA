
%%%%%%%%%%%%%%%%%%%%%%%%%%%%%%%%%%%%%%%%%%%%%%%%%%%%%%%%%%%%%%%%%%%%%%%%%%%%%%%%%%%%%%%%%%%%%
\begin{frame}[allowframebreaks,c]{} \label{SCF}

\begin{center}
\Huge \textbf{SCF}
\end{center}

\end{frame}
%%%%%%%%%%%%%%%%%%%%%%%%%%%%%%%%%%%%%%%%%%%%%%%%%%%%%%%%%%%%%%%%%%%%%%%%%%%%%%%%%%%%%%%%%%%%%


%%%%%%%%%%%%%%%%%%%%%%%%%%%%%%%%%%%%%%%%%%%%%%%%%%%%%%%%%%%%%%%%%%%%%%%%%%%%%%%%%%%%%%%%%%%%%
\begin{frame}[allowframebreaks]{\texttt{CHEB\_DEGREE}} \label{CHEB_DEGREE}
\vspace*{-12pt}
\begin{columns}
\column{0.4\linewidth}
\begin{block}{Type}
Integer
\end{block}

\begin{block}{Default}
Automatically set.
\end{block}

\column{0.4\linewidth}
\begin{block}{Unit}
No unit
\end{block}

\begin{block}{Example}
\texttt{CHEB\_DEGEE}: 25
\end{block}
\end{columns}

\begin{block}{Description}
Degree of polynomial used for Chebyshev filtering. 
\end{block}

\begin{block}{Remark}
For larger mesh-sizes, smaller values of \texttt{CHEB\_DEGREE} are generally more efficient, and vice-versa.
\end{block}

\end{frame}
%%%%%%%%%%%%%%%%%%%%%%%%%%%%%%%%%%%%%%%%%%%%%%%%%%%%%%%%%%%%%%%%%%%%%%%%%%%%%%%%%%%%%%%%%%%%%


%%%%%%%%%%%%%%%%%%%%%%%%%%%%%%%%%%%%%%%%%%%%%%%%%%%%%%%%%%%%%%%%%%%%%%%%%%%%%%%%%%%%%%%%%%%%%
\begin{frame}[allowframebreaks]{\texttt{CHEFSI\_BOUND\_FLAG}} \label{CHEFSI_BOUND_FLAG}
\vspace*{-12pt}
\begin{columns}
\column{0.4\linewidth}
\begin{block}{Type}
Integer
\end{block}

\begin{block}{Default}
0
\end{block}

\column{0.4\linewidth}
\begin{block}{Unit}
No unit
\end{block}

\begin{block}{Example}
\texttt{CHEFSI\_BOUND\_FLAG}: 1
\end{block}
\end{columns}

\begin{block}{Description}
Flag to recalculate the bounds for Chebyshev filtering. If set to $0$, then only for the very first SCF will the upper bound be evaluated based on the maximum eigenvalue using Lanczos algorithm, and the upper bound will be the same for the rest steps. If set to $1$, the upper bound will be reevaluated for every SCF.
\end{block}

\end{frame}
%%%%%%%%%%%%%%%%%%%%%%%%%%%%%%%%%%%%%%%%%%%%%%%%%%%%%%%%%%%%%%%%%%%%%%%%%%%%%%%%%%%%%%%%%%%%%


%%%%%%%%%%%%%%%%%%%%%%%%%%%%%%%%%%%%%%%%%%%%%%%%%%%%%%%%%%%%%%%%%%%%%%%%%%%%%%%%%%%%%%%%%%%%%
\begin{frame}[allowframebreaks]{\texttt{RHO\_TRIGGER}} \label{RHO_TRIGGER}
\vspace*{-12pt}
\begin{columns}
\column{0.4\linewidth}
\begin{block}{Type}
Integer
\end{block}

\begin{block}{Default}
4
\end{block}

\column{0.4\linewidth}
\begin{block}{Unit}
No unit
\end{block}

\begin{block}{Example}
\texttt{RHO\_TRIGGER}: 5
\end{block}
\end{columns}

\begin{block}{Description}
The number of times Chebyshev filtering is repeated before updating the electron density in the very first SCF iteration.
\end{block}

\begin{block}{Remark}
Values smaller than the default value of 4 can result in a significant increase in the number of SCF
iterations. Larger values can sometimes reduce the number of SCF iterations. 
\end{block}

\end{frame}
%%%%%%%%%%%%%%%%%%%%%%%%%%%%%%%%%%%%%%%%%%%%%%%%%%%%%%%%%%%%%%%%%%%%%%%%%%%%%%%%%%%%%%%%%%%%%




%%%%%%%%%%%%%%%%%%%%%%%%%%%%%%%%%%%%%%%%%%%%%%%%%%%%%%%%%%%%%%%%%%%%%%%%%%%%%%%%%%%%%%%%%%%%%
\begin{frame}[allowframebreaks]{\texttt{{MAXIT\_SCF}}} \label{MAXIT_SCF}
\vspace*{-12pt}
\begin{columns}
\column{0.4\linewidth}
\begin{block}{Type}
Integer
\end{block}

\begin{block}{Default}
100
\end{block}

\column{0.4\linewidth}
\begin{block}{Unit}
No unit
\end{block}

\begin{block}{Example}
\texttt{MAXIT\_SCF}: 50
\end{block}
\end{columns}

\begin{block}{Description}
Maximum number of SCF iterations.
\end{block}

\begin{block}{Remark}
Larger values than the default of 100 may be required for highly inhomogeneous systems, particularly when small values of \hyperlink{SMEARING}{\texttt{SMEARING}}/\hyperlink{ELEC_TEMP}{\texttt{ELEC\_TEMP}} are chosen. 
\end{block}

\end{frame}
%%%%%%%%%%%%%%%%%%%%%%%%%%%%%%%%%%%%%%%%%%%%%%%%%%%%%%%%%%%%%%%%%%%%%%%%%%%%%%%%%%%%%%%%%%%%%


%%%%%%%%%%%%%%%%%%%%%%%%%%%%%%%%%%%%%%%%%%%%%%%%%%%%%%%%%%%%%%%%%%%%%%%%%%%%%%%%%%%%%%%%%%%%%
\begin{frame}[allowframebreaks]{\texttt{{MINIT\_SCF}}} \label{MINIT_SCF}
\vspace*{-12pt}
\begin{columns}
\column{0.4\linewidth}
\begin{block}{Type}
Integer
\end{block}

\begin{block}{Default}
3
\end{block}

\column{0.4\linewidth}
\begin{block}{Unit}
No unit
\end{block}

\begin{block}{Example}
\texttt{MAXIT\_SCF}: 5
\end{block}
\end{columns}

\begin{block}{Description}
Minimum number of SCF iterations.
\end{block}

\end{frame}
%%%%%%%%%%%%%%%%%%%%%%%%%%%%%%%%%%%%%%%%%%%%%%%%%%%%%%%%%%%%%%%%%%%%%%%%%%%%%%%%%%%%%%%%%%%%%


%%%%%%%%%%%%%%%%%%%%%%%%%%%%%%%%%%%%%%%%%%%%%%%%%%%%%%%%%%%%%%%%%%%%%%%%%%%%%%%%%%%%%%%%%%%%%
\begin{frame}[allowframebreaks]{\texttt{TOL\_SCF}} \label{TOL_SCF}
\vspace*{-12pt}
\begin{columns}
\column{0.4\linewidth}
\begin{block}{Type}
Double
\end{block}

\begin{block}{Default}
Automatically set for $10^{-3}$ Ha/Bohr accuracy in force
\end{block}

\column{0.4\linewidth}
\begin{block}{Unit}
No unit
\end{block}

\begin{block}{Example}
\texttt{TOL\_SCF}: 1e-5
\end{block}
\end{columns}

\begin{block}{Description}
The tolerance on the normalized residual of the effective potential or the electron density for convergence of the SCF iteration. 
\end{block}

\begin{block}{Remark}
Only one of \hyperlink{TOL_SCF}{\texttt{TOL\_SCF}}, \hyperlink{SCF_ENERGY_ACC}{\texttt{SCF\_ENERGY\_ACC}}, or \hyperlink{SCF_FORCE_ACC}{\texttt{SCF\_FORCE\_ACC}} can be specified.
\end{block}

\end{frame}
%%%%%%%%%%%%%%%%%%%%%%%%%%%%%%%%%%%%%%%%%%%%%%%%%%%%%%%%%%%%%%%%%%%%%%%%%%%%%%%%%%%%%%%%%%%%%




%%%%%%%%%%%%%%%%%%%%%%%%%%%%%%%%%%%%%%%%%%%%%%%%%%%%%%%%%%%%%%%%%%%%%%%%%%%%%%%%%%%%%%%%%%%%%
\begin{frame}[allowframebreaks]{\texttt{SCF\_FORCE\_ACC}} \label{SCF_FORCE_ACC}
\vspace*{-12pt}
\begin{columns}
\column{0.4\linewidth}
\begin{block}{Type}
Double
\end{block}

\begin{block}{Default}
None
\end{block}

\column{0.4\linewidth}
\begin{block}{Unit}
Ha/Bohr
\end{block}

\begin{block}{Example}
\texttt{SCF\_FORCE\_ACC}: 1e-4
\end{block}
\end{columns}

\begin{block}{Description}
The tolerance on the atomic forces for convergence of the SCF iteration. 
\end{block}

\begin{block}{Remark}
Only one of \hyperlink{SCF_FORCE_ACC}{\texttt{SCF\_FORCE\_ACC}}, \hyperlink{TOL_SCF}{\texttt{TOL\_SCF}} or \hyperlink{SCF_ENERGY_ACC}{\texttt{SCF\_ENERGY\_ACC}} can be specified.
\end{block}

\end{frame}
%%%%%%%%%%%%%%%%%%%%%%%%%%%%%%%%%%%%%%%%%%%%%%%%%%%%%%%%%%%%%%%%%%%%%%%%%%%%%%%%%%%%%%%%%%%%%


%%%%%%%%%%%%%%%%%%%%%%%%%%%%%%%%%%%%%%%%%%%%%%%%%%%%%%%%%%%%%%%%%%%%%%%%%%%%%%%%%%%%%%%%%%%%%
\begin{frame}[allowframebreaks]{\texttt{SCF\_ENERGY\_ACC}} \label{SCF_ENERGY_ACC}
\vspace*{-12pt}
\begin{columns}
\column{0.4\linewidth}
\begin{block}{Type}
Double
\end{block}

\begin{block}{Default}
None
\end{block}

\column{0.4\linewidth}
\begin{block}{Unit}
Ha/atom
\end{block}

\begin{block}{Example}
\texttt{SCF\_ENERGY\_ACC}: 1e-5
\end{block}
\end{columns}

\begin{block}{Description}
The tolerance on the free energy for the convergence of the SCF iteration. 
\end{block}

\begin{block}{Remark}
Only one of \hyperlink{SCF_ENERGY_ACC}{\texttt{SCF\_ENERGY\_ACC}}, \hyperlink{SCF_FORCE_ACC}{\texttt{SCF\_FORCE\_ACC}}, or \hyperlink{TOL_SCF}{\texttt{TOL\_SCF}} can be specified.
\end{block}

\end{frame}
%%%%%%%%%%%%%%%%%%%%%%%%%%%%%%%%%%%%%%%%%%%%%%%%%%%%%%%%%%%%%%%%%%%%%%%%%%%%%%%%%%%%%%%%%%%%%




%%%%%%%%%%%%%%%%%%%%%%%%%%%%%%%%%%%%%%%%%%%%%%%%%%%%%%%%%%%%%%%%%%%%%%%%%%%%%%%%%%%%%%%%%%%%%
\begin{frame}[allowframebreaks]{\texttt{TOL\_LANCZOS}} \label{TOL_LANCZOS}
\vspace*{-12pt}
\begin{columns}
\column{0.4\linewidth}
\begin{block}{Type}
Double
\end{block}

\begin{block}{Default}
1e-2
\end{block}

\column{0.4\linewidth}
\begin{block}{Unit}
No unit
\end{block}

\begin{block}{Example}
\texttt{TOL\_LANCZOS}: 1e-3
\end{block}
\end{columns}

\begin{block}{Description}
The tolerance within the Lanczos algorithm for calculating the extremal eigenvalues of the Hamiltonian, required as part of the CheFSI method. 
\end{block}

\begin{block}{Remark}
Typically, the Lanczos tolerance does not need to be very strict.
\end{block}

\end{frame}
%%%%%%%%%%%%%%%%%%%%%%%%%%%%%%%%%%%%%%%%%%%%%%%%%%%%%%%%%%%%%%%%%%%%%%%%%%%%%%%%%%%%%%%%%%%%%




%%%%%%%%%%%%%%%%%%%%%%%%%%%%%%%%%%%%%%%%%%%%%%%%%%%%%%%%%%%%%%%%%%%%%%%%%%%%%%%%%%%%%%%%%%%%%
\begin{frame}[allowframebreaks]{\texttt{MIXING\_VARIABLE}} \label{MIXING_VARIABLE}
\vspace*{-12pt}
\begin{columns}
\column{0.4\linewidth}
\begin{block}{Type}
String
\end{block}

\begin{block}{Default}
\texttt{density}
\end{block}

\column{0.4\linewidth}
\begin{block}{Unit}
No unit
\end{block}

\begin{block}{Example}
\texttt{MIXING\_VARIABLE}: \texttt{potential}
\end{block}
\end{columns}

\begin{block}{Description}
This specifies whether potential or density mixing is performed in the SCF iteration. Available options are: \texttt{potential} and \texttt{density}.
\end{block}

\end{frame}
%%%%%%%%%%%%%%%%%%%%%%%%%%%%%%%%%%%%%%%%%%%%%%%%%%%%%%%%%%%%%%%%%%%%%%%%%%%%%%%%%%%%%%%%%%%%%



%%%%%%%%%%%%%%%%%%%%%%%%%%%%%%%%%%%%%%%%%%%%%%%%%%%%%%%%%%%%%%%%%%%%%%%%%%%%%%%%%%%%%%%%%%%%%
\begin{frame}[allowframebreaks]{\texttt{MIXING\_HISTORY}} \label{MIXING_HISTORY}
\vspace*{-12pt}
\begin{columns}
\column{0.4\linewidth}
\begin{block}{Type}
Integer
\end{block}

\begin{block}{Default}
7
\end{block}

\column{0.4\linewidth}
\begin{block}{Unit}
No unit
\end{block}

\begin{block}{Example}
\texttt{MIXING\_HISTORY}: 40
\end{block}
\end{columns}

\begin{block}{Description}
The mixing history used in Pulay mixing.
\end{block}

\begin{block}{Remark}
Too small values of \hyperlink{MIXING_HISTORY}{\texttt{MIXING\_HISTORY}} can result in poor SCF convergence.
\end{block}

\end{frame}
%%%%%%%%%%%%%%%%%%%%%%%%%%%%%%%%%%%%%%%%%%%%%%%%%%%%%%%%%%%%%%%%%%%%%%%%%%%%%%%%%%%%%%%%%%%%%



%%%%%%%%%%%%%%%%%%%%%%%%%%%%%%%%%%%%%%%%%%%%%%%%%%%%%%%%%%%%%%%%%%%%%%%%%%%%%%%%%%%%%%%%%%%%%
\begin{frame}[allowframebreaks]{\texttt{MIXING\_PARAMETER}} \label{MIXING_PARAMETER}
\vspace*{-12pt}
\begin{columns}
\column{0.4\linewidth}
\begin{block}{Type}
Double
\end{block}

\begin{block}{Default}
0.3
\end{block}

\column{0.4\linewidth}
\begin{block}{Unit}
No unit
\end{block}

\begin{block}{Example}
\texttt{MIXING\_PARAMETER}: 0.1
\end{block}
\end{columns}

\begin{block}{Description}
The value of the relaxation parameter used in Pulay/simple mixing.
\end{block}

\begin{block}{Remark}
Values larger than the default value of 0.3 can be used for insulating systems, whereas smaller values are generally required for metallic  systems, particularly at small values of \hyperlink{SMEARING}{\texttt{SMEARING}} or \hyperlink{ELEC_TEMP}{\texttt{ELEC\_TEMP}}.
\end{block}

\end{frame}
%%%%%%%%%%%%%%%%%%%%%%%%%%%%%%%%%%%%%%%%%%%%%%%%%%%%%%%%%%%%%%%%%%%%%%%%%%%%%%%%%%%%%%%%%%%%%


%%%%%%%%%%%%%%%%%%%%%%%%%%%%%%%%%%%%%%%%%%%%%%%%%%%%%%%%%%%%%%%%%%%%%%%%%%%%%%%%%%%%%%%%%%%%%
\begin{frame}[allowframebreaks]{\texttt{MIXING\_PARAMETER\_SIMPLE}} \label{MIXING_PARAMETER_SIMPLE}
\vspace*{-12pt}
\begin{columns}
\column{0.4\linewidth}
\begin{block}{Type}
Double
\end{block}

\begin{block}{Default}
Automatically set to the same as \hyperlink{MIXING_PARAMETER}{\texttt{MIXING\_PARAMETER}}
\end{block}

\column{0.4\linewidth}
\begin{block}{Unit}
No unit
\end{block}

\begin{block}{Example}
\texttt{MIXING\_PARAMETER\_SIMPLE}: 0.1
\end{block}
\end{columns}

\begin{block}{Description}
The value of the relaxation parameter used in the simple mixing step in the periodic Pulay scheme.
\end{block}

\end{frame}
%%%%%%%%%%%%%%%%%%%%%%%%%%%%%%%%%%%%%%%%%%%%%%%%%%%%%%%%%%%%%%%%%%%%%%%%%%%%%%%%%%%%%%%%%%%%%


%%%%%%%%%%%%%%%%%%%%%%%%%%%%%%%%%%%%%%%%%%%%%%%%%%%%%%%%%%%%%%%%%%%%%%%%%%%%%%%%%%%%%%%%%%%%%
\begin{frame}[allowframebreaks]{\texttt{MIXING\_PARAMETER\_MAG}} \label{MIXING_PARAMETER_MAG}
\vspace*{-12pt}
\begin{columns}
\column{0.4\linewidth}
\begin{block}{Type}
Double
\end{block}

\begin{block}{Default}
Automatically set to the same as \hyperlink{MIXING_PARAMETER}{\texttt{MIXING\_PARAMETER}}.
\end{block}

\column{0.4\linewidth}
\begin{block}{Unit}
No unit
\end{block}

\begin{block}{Example}
\texttt{MIXING\_PARAMETER\_MAG}: 4.0
\end{block}
\end{columns}

\begin{block}{Description}
The mixing parameter for the magnetization density in Pulay mixing for spin-polarized calculations.
\end{block}

\begin{block}{Remark}
    For spin-polarized calculations, when SCF has difficulty to converge, increasing the mixing parameter to magnetization density might help. For example, setting it to 4.0, while turning off the preconditioner applied to the magnetization density (by setting \hyperlink{MIXING_PRECOND_MAG}{\texttt{MIXING\_PRECOND\_MAG}} to `\texttt{none}') is a good choice.
\end{block}

\end{frame}
%%%%%%%%%%%%%%%%%%%%%%%%%%%%%%%%%%%%%%%%%%%%%%%%%%%%%%%%%%%%%%%%%%%%%%%%%%%%%%%%%%%%%%%%%%%%%


%%%%%%%%%%%%%%%%%%%%%%%%%%%%%%%%%%%%%%%%%%%%%%%%%%%%%%%%%%%%%%%%%%%%%%%%%%%%%%%%%%%%%%%%%%%%%
\begin{frame}[allowframebreaks]{\texttt{MIXING\_PARAMETER\_SIMPLE\_MAG}} \label{MIXING_PARAMETER_SIMPLE_MAG}
\vspace*{-12pt}
\begin{columns}
\column{0.4\linewidth}
\begin{block}{Type}
Double
\end{block}

\begin{block}{Default}
Automatically set to the same as \hyperlink{MIXING_PARAMETER_MAG}{\texttt{MIXING\_PARAMETER\_MAG}}
\end{block}

\column{0.5\linewidth}
\begin{block}{Unit}
No unit
\end{block}

\begin{block}{Example}
\texttt{MIXING\_PARAMETER\_SIMPLE\_MAG}: 4.0
\end{block}
\end{columns}

\begin{block}{Description}
The value of the relaxation parameter for the magnetization density used in the simple mixing step in the periodic Pulay scheme for spin-polarized calculations.
\end{block}

\end{frame}
%%%%%%%%%%%%%%%%%%%%%%%%%%%%%%%%%%%%%%%%%%%%%%%%%%%%%%%%%%%%%%%%%%%%%%%%%%%%%%%%%%%%%%%%%%%%%



%%%%%%%%%%%%%%%%%%%%%%%%%%%%%%%%%%%%%%%%%%%%%%%%%%%%%%%%%%%%%%%%%%%%%%%%%%%%%%%%%%%%%%%%%%%%%
\begin{frame}[allowframebreaks]{\texttt{PULAY\_FREQUENCY}} \label{PULAY_FREQUENCY}
\vspace*{-12pt}
\begin{columns}
\column{0.4\linewidth}
\begin{block}{Type}
Integer
\end{block}

\begin{block}{Default}
1
\end{block}

\column{0.4\linewidth}
\begin{block}{Unit}
No unit
\end{block}

\begin{block}{Example}
\texttt{PULAY\_FREQUENCY}: 4
\end{block}
\end{columns}

\begin{block}{Description}
The frequency of Pulay mixing in Periodic Pulay. 
\end{block}

\begin{block}{Remark}
The default value of 1 corresponds to Pulay mixing.
\end{block}

\end{frame}
%%%%%%%%%%%%%%%%%%%%%%%%%%%%%%%%%%%%%%%%%%%%%%%%%%%%%%%%%%%%%%%%%%%%%%%%%%%%%%%%%%%%%%%%%%%%%




%%%%%%%%%%%%%%%%%%%%%%%%%%%%%%%%%%%%%%%%%%%%%%%%%%%%%%%%%%%%%%%%%%%%%%%%%%%%%%%%%%%%%%%%%%%%%
\begin{frame}[allowframebreaks]{\texttt{PULAY\_RESTART}} \label{PULAY_RESTART}
\vspace*{-12pt}
\begin{columns}
\column{0.4\linewidth}
\begin{block}{Type}
Integer
\end{block}

\begin{block}{Default}
0
\end{block}

\column{0.5\linewidth}
\begin{block}{Unit}
No unit
\end{block}

\begin{block}{Example}
\texttt{PULAY\_RESTART}: 1
\end{block}
\end{columns}

\begin{block}{Description}
The flag for restarting the `Periodic Pulay' mixing. If set to 0, the restarted Pulay method is turned off.
\end{block}

\end{frame}
%%%%%%%%%%%%%%%%%%%%%%%%%%%%%%%%%%%%%%%%%%%%%%%%%%%%%%%%%%%%%%%%%%%%%%%%%%%%%%%%%%%%%%%%%%%%%



%%%%%%%%%%%%%%%%%%%%%%%%%%%%%%%%%%%%%%%%%%%%%%%%%%%%%%%%%%%%%%%%%%%%%%%%%%%%%%%%%%%%%%%%%%%%%
\begin{frame}[allowframebreaks]{\texttt{MIXING\_PRECOND}} \label{MIXING_PRECOND}
\vspace*{-12pt}
\begin{columns}
\column{0.4\linewidth}
\begin{block}{Type}
String
\end{block}

\begin{block}{Default}
\texttt{kerker}
\end{block}

\column{0.4\linewidth}
\begin{block}{Unit}
No unit
\end{block}

\begin{block}{Example}
\texttt{MIXING\_PRECOND}: \texttt{none}
\end{block}
\end{columns}

\begin{block}{Description}
This specifies the preconditioner used in the SCF iteration. Available options are: \texttt{none}, \texttt{kerker}.
\end{block}

\end{frame}
%%%%%%%%%%%%%%%%%%%%%%%%%%%%%%%%%%%%%%%%%%%%%%%%%%%%%%%%%%%%%%%%%%%%%%%%%%%%%%%%%%%%%%%%%%%%%


%%%%%%%%%%%%%%%%%%%%%%%%%%%%%%%%%%%%%%%%%%%%%%%%%%%%%%%%%%%%%%%%%%%%%%%%%%%%%%%%%%%%%%%%%%%%%
\begin{frame}[allowframebreaks]{\texttt{MIXING\_PRECOND\_MAG}} \label{MIXING_PRECOND_MAG}
\vspace*{-12pt}
\begin{columns}
\column{0.4\linewidth}
\begin{block}{Type}
String
\end{block}

\begin{block}{Default}
\texttt{none}
\end{block}

\column{0.5\linewidth}
\begin{block}{Unit}
No unit
\end{block}

\begin{block}{Example}
\texttt{MIXING\_PRECOND\_MAG}: \texttt{kerker}
\end{block}
\end{columns}

\begin{block}{Description}
This specifies the preconditioner used for the magnetization density in the SCF iteration for spin-polarized calculations. Available options are: \texttt{none}, \texttt{kerker}.
\end{block}

\end{frame}
%%%%%%%%%%%%%%%%%%%%%%%%%%%%%%%%%%%%%%%%%%%%%%%%%%%%%%%%%%%%%%%%%%%%%%%%%%%%%%%%%%%%%%%%%%%%%



%%%%%%%%%%%%%%%%%%%%%%%%%%%%%%%%%%%%%%%%%%%%%%%%%%%%%%%%%%%%%%%%%%%%%%%%%%%%%%%%%%%%%%%%%%%%%
\begin{frame}[allowframebreaks]{\texttt{TOL\_PRECOND}} \label{TOL_PRECOND}
\vspace*{-12pt}
\begin{columns}
\column{0.4\linewidth}
\begin{block}{Type}
Double
\end{block}

\begin{block}{Default}
$h^2\times0.001$
\end{block}

\column{0.4\linewidth}
\begin{block}{Unit}
No unit
\end{block}

\begin{block}{Example}
\texttt{TOL\_PRECOND}: 1e-4
\end{block}
\end{columns}

\begin{block}{Description}
The tolerance on the relative residual for the linear systems arising during the real-space preconditioning of the SCF.
\end{block}

\begin{block}{Remark}
The linear systems do not need to be solved very accurately. $h$ is the mesh spacing.
\end{block}

\end{frame}
%%%%%%%%%%%%%%%%%%%%%%%%%%%%%%%%%%%%%%%%%%%%%%%%%%%%%%%%%%%%%%%%%%%%%%%%%%%%%%%%%%%%%%%%%%%%%



%%%%%%%%%%%%%%%%%%%%%%%%%%%%%%%%%%%%%%%%%%%%%%%%%%%%%%%%%%%%%%%%%%%%%%%%%%%%%%%%%%%%%%%%%%%%%
\begin{frame}[allowframebreaks]{\texttt{PRECOND\_KERKER\_KTF}} \label{PRECOND_KERKER_KTF}
\vspace*{-12pt}
\begin{columns}
\column{0.4\linewidth}
\begin{block}{Type}
Double
\end{block}

\begin{block}{Default}
1.0
\end{block}

\column{0.5\linewidth}
\begin{block}{Unit}
$\textrm{Bohr}^{-1}$
\end{block}

\begin{block}{Example}
\texttt{PRECOND\_KERKER\_KTF}: 0.8
\end{block}
\end{columns}

\begin{block}{Description}
The Thomas-Fermi screening length appearing in the \texttt{kerker} preconditioner (\hyperlink{MIXING_PRECOND}{\texttt{MIXING\_PRECOND}}). 
\end{block}

\end{frame}
%%%%%%%%%%%%%%%%%%%%%%%%%%%%%%%%%%%%%%%%%%%%%%%%%%%%%%%%%%%%%%%%%%%%%%%%%%%%%%%%%%%%%%%%%%%%%



%%%%%%%%%%%%%%%%%%%%%%%%%%%%%%%%%%%%%%%%%%%%%%%%%%%%%%%%%%%%%%%%%%%%%%%%%%%%%%%%%%%%%%%%%%%%%
\begin{frame}[allowframebreaks]{\texttt{PRECOND\_KERKER\_THRESH}} \label{PRECOND_KERKER_THRESH}
\vspace*{-12pt}
\begin{columns}
\column{0.4\linewidth}
\begin{block}{Type}
Double
\end{block}

\begin{block}{Default}
0.1
\end{block}

\column{0.5\linewidth}
\begin{block}{Unit}
No unit
\end{block}

\begin{block}{Example}
\texttt{PRECOND\_KERKER\_THRESH}: 0.0
\end{block}
\end{columns}

\begin{block}{Description}
The threshold for the \texttt{kerker} preconditioner (\hyperlink{MIXING_PRECOND}{\texttt{MIXING\_PRECOND}}).
\end{block}

\begin{block}{Remark}
This threshold will be scaled by the \hyperlink{MIXING_PARAMETER}{\texttt{MIXING\_PARAMETER}}. If the threshold is set to 0, the original \texttt{kerker} preconditioner is recovered.
\end{block}

\end{frame}
%%%%%%%%%%%%%%%%%%%%%%%%%%%%%%%%%%%%%%%%%%%%%%%%%%%%%%%%%%%%%%%%%%%%%%%%%%%%%%%%%%%%%%%%%%%%%


%%%%%%%%%%%%%%%%%%%%%%%%%%%%%%%%%%%%%%%%%%%%%%%%%%%%%%%%%%%%%%%%%%%%%%%%%%%%%%%%%%%%%%%%%%%%%
\begin{frame}[allowframebreaks]{\texttt{PRECOND\_KERKER\_KTF\_MAG}} \label{PRECOND_KERKER_KTF_MAG}
\vspace*{-12pt}
\begin{columns}
\column{0.4\linewidth}
\begin{block}{Type}
Double
\end{block}

\begin{block}{Default}
1.0
\end{block}

\column{0.5\linewidth}
\begin{block}{Unit}
$\textrm{Bohr}^{-1}$
\end{block}

\begin{block}{Example}
\texttt{PRECOND\_KERKER\_KTF\_MAG}: 0.8
\end{block}
\end{columns}

\begin{block}{Description}
The Thomas-Fermi screening length appearing in the \texttt{kerker} preconditioner for the magnetization density (\hyperlink{MIXING_PRECOND_MAG}{\texttt{MIXING\_PRECOND\_MAG}}). 
\end{block}

\end{frame}
%%%%%%%%%%%%%%%%%%%%%%%%%%%%%%%%%%%%%%%%%%%%%%%%%%%%%%%%%%%%%%%%%%%%%%%%%%%%%%%%%%%%%%%%%%%%%



%%%%%%%%%%%%%%%%%%%%%%%%%%%%%%%%%%%%%%%%%%%%%%%%%%%%%%%%%%%%%%%%%%%%%%%%%%%%%%%%%%%%%%%%%%%%%
\begin{frame}[allowframebreaks]{\texttt{PRECOND\_KERKER\_THRESH\_MAG}} \label{PRECOND_KERKER_THRESH_MAG}
\vspace*{-12pt}
\begin{columns}
\column{0.4\linewidth}
\begin{block}{Type}
Double
\end{block}

\begin{block}{Default}
0.1
\end{block}

\column{0.5\linewidth}
\begin{block}{Unit}
No unit
\end{block}

\begin{block}{Example}
\texttt{PRECOND\_KERKER\_THRESH\_MAG}: 0.0
\end{block}
\end{columns}

\begin{block}{Description}
The threshold for the \texttt{kerker} preconditioner the magnetization density (\hyperlink{MIXING_PRECOND_MAG}{\texttt{MIXING\_PRECOND\_MAG}}). 
\end{block}

\begin{block}{Remark}
This threshold will be scaled by the \hyperlink{MIXING_PARAMETER_MAG}{\texttt{MIXING\_PARAMETER\_MAG}}. If the threshold is set to 0, the original \texttt{kerker} preconditioner is recovered.
\end{block}

\end{frame}
%%%%%%%%%%%%%%%%%%%%%%%%%%%%%%%%%%%%%%%%%%%%%%%%%%%%%%%%%%%%%%%%%%%%%%%%%%%%%%%%%%%%%%%%%%%%%





%%%%%%%%%%%%%%%%%%%%%%%%%%%%%%%%%%%%%%%%%%%%%%%%%%%%%%%%%%%%%%%%%%%%%%%%%%%%%%%%%%%%%%%%%%%%%
%\begin{frame}[allowframebreaks]{\texttt{PRECOND\_RESTA\_Q0}} \label{PRECOND_RESTA_Q0}
%\vspace*{-12pt}
%\begin{columns}
%\column{0.4\linewidth}
%\begin{block}{Type}
%Double
%\end{block}
%
%\begin{block}{Default}
%1.36
%\end{block}
%
%\column{0.5\linewidth}
%\begin{block}{Unit}
%$\textrm{Bohr}^{-1}$
%\end{block}
%
%\begin{block}{Example}
%\texttt{PRECOND\_RESTA\_Q0}: 1.10
%\end{block}
%\end{columns}
%
%\begin{block}{Description}
%The Fermi-momentum-related quantity appearing in \texttt{resta} preconditioner (\hyperlink{MIXING_PRECOND}{\texttt{MIXING\_PRECOND}}).
%\end{block}
%
%\end{frame}
%%%%%%%%%%%%%%%%%%%%%%%%%%%%%%%%%%%%%%%%%%%%%%%%%%%%%%%%%%%%%%%%%%%%%%%%%%%%%%%%%%%%%%%%%%%%%



%%%%%%%%%%%%%%%%%%%%%%%%%%%%%%%%%%%%%%%%%%%%%%%%%%%%%%%%%%%%%%%%%%%%%%%%%%%%%%%%%%%%%%%%%%%%%
%\begin{frame}[allowframebreaks]{\texttt{PRECOND\_RESTA\_RS}} \label{PRECOND_RESTA_RS}
%\vspace*{-12pt}
%\begin{columns}
%\column{0.4\linewidth}
%\begin{block}{Type}
%Double
%\end{block}
%
%\begin{block}{Default}
%2.76
%\end{block}
%
%\column{0.5\linewidth}
%\begin{block}{Unit}
%Bohr
%\end{block}
%
%\begin{block}{Example}
%\texttt{PRECOND\_RESTA\_RS}: 4.28
%\end{block}
%\end{columns}
%
%\begin{block}{Description}
%The screening length appearing in the \texttt{resta} preconditioner (\hyperlink{MIXING_PRECOND}{\texttt{MIXING\_PRECOND}}). 
%\end{block}
%
%\end{frame}
%%%%%%%%%%%%%%%%%%%%%%%%%%%%%%%%%%%%%%%%%%%%%%%%%%%%%%%%%%%%%%%%%%%%%%%%%%%%%%%%%%%%%%%%%%%%%



%%%%%%%%%%%%%%%%%%%%%%%%%%%%%%%%%%%%%%%%%%%%%%%%%%%%%%%%%%%%%%%%%%%%%%%%%%%%%%%%%%%%%%%%%%%%%
%\begin{frame}[allowframebreaks]{\texttt{PRECOND\_FITPOW}} \label{PRECOND_FITPOW}
%\vspace*{-12pt}
%\begin{columns}
%\column{0.4\linewidth}
%\begin{block}{Type}
%Integer
%\end{block}
%
%\begin{block}{Default}
%2
%\end{block}
%
%\column{0.4\linewidth}
%\begin{block}{Unit}
%No unit
%\end{block}
%
%\begin{block}{Example}
%\texttt{PRECOND\_FITPOW}: 3
%\end{block}
%\end{columns}
%
%\begin{block}{Description}
%Half of the highest degree of rational polynomials used for the real-space preconditioning of the SCF iteration. 
%\end{block}
%
%\begin{block}{Remark}
%Currently this number cannot be larger than 5. Used only for the \texttt{resta} and \texttt{truncated\_kerker} preconditioners.
%\end{block}
%
%\end{frame}


%%%%%%%%%%%%%%%%%%%%%%%%%%%%%%%%%%%%%%%%%%%%%%%%%%%%%%%%%%%%%%%%%%%%%%%%%%%%%%%%%%%%%%%%%%%%%
\begin{frame}[allowframebreaks]{\texttt{FIX\_RAND}} \label{FIX_RAND}
\vspace*{-12pt}
\begin{columns}
\column{0.4\linewidth}
\begin{block}{Type}
Integer
\end{block}

\begin{block}{Default}
0
\end{block}

\column{0.4\linewidth}
\begin{block}{Unit}
No unit
\end{block}

\begin{block}{Example}
\texttt{FIX\_RAND}: \texttt{1}
\end{block}
\end{columns}

\begin{block}{Description}
Flag to fix the random seeds for setting initial guesses. Once set to $1$, the random seeds will be fixed for different runs and for different numbers of processors. This option will make sure the answers will be exactly the same (up to machine precision) when SPARC is executed with different numbers of processors.
\end{block}

\end{frame}
%%%%%%%%%%%%%%%%%%%%%%%%%%%%%%%%%%%%%%%%%%%%%%%%%%%%%%%%%%%%%%%%%%%%%%%%%%%%%%%%%%%%%%%%%%%%%


%%%%%%%%%%%%%%%%%%%%%%%%%%%%%%%%%%%%%%%%%%%%%%%%%%%%%%%%%%%%%%%%%%%%%%%%%%%%%%%%%%%%%%%%%%%%%
\begin{frame}[allowframebreaks]{\texttt{TOL\_FOCK}} \label{TOL_FOCK}
\vspace*{-12pt}
\begin{columns}
\column{0.4\linewidth}
\begin{block}{Type}
Double
\end{block}

\begin{block}{Default}
$0.2*$\hyperlink{TOL_SCF}{\texttt{TOL\_SCF}}
\end{block}

\column{0.4\linewidth}
\begin{block}{Unit}
No unit
\end{block}

\begin{block}{Example}
\texttt{TOL\_FOCK}: 1e-6
\end{block}
\end{columns}

\begin{block}{Description}
The tolerance on the Hartree-Fock outer loop, measured by the exact exchange energy difference per atom in 2 consecutive outer loops.
\end{block}

\begin{block}{Remark}
Only active when using hybrid functionals, like PBE0 and HSE. 
\end{block}

\end{frame}
%%%%%%%%%%%%%%%%%%%%%%%%%%%%%%%%%%%%%%%%%%%%%%%%%%%%%%%%%%%%%%%%%%%%%%%%%%%%%%%%%%%%%%%%%%%%%



%%%%%%%%%%%%%%%%%%%%%%%%%%%%%%%%%%%%%%%%%%%%%%%%%%%%%%%%%%%%%%%%%%%%%%%%%%%%%%%%%%%%%%%%%%%%%
\begin{frame}[allowframebreaks]{\texttt{MAXIT\_FOCK}} \label{MAXIT_FOCK}
\vspace*{-12pt}
\begin{columns}
\column{0.4\linewidth}
\begin{block}{Type}
Integer
\end{block}

\begin{block}{Default}
20
\end{block}

\column{0.4\linewidth}
\begin{block}{Unit}
No unit
\end{block}

\begin{block}{Example}
\texttt{MAXIT\_FOCK}: 50
\end{block}
\end{columns}

\begin{block}{Description}
The maximum number of iterations for Hartree-Fock outer loop.
\end{block}

\begin{block}{Remark}
Only active when using hybrid functionals, like PBE0 and HSE. 
\end{block}

\end{frame}
%%%%%%%%%%%%%%%%%%%%%%%%%%%%%%%%%%%%%%%%%%%%%%%%%%%%%%%%%%%%%%%%%%%%%%%%%%%%%%%%%%%%%%%%%%%%%

%%%%%%%%%%%%%%%%%%%%%%%%%%%%%%%%%%%%%%%%%%%%%%%%%%%%%%%%%%%%%%%%%%%%%%%%%%%%%%%%%%%%%%%%%%%%%
\begin{frame}[allowframebreaks]{\texttt{MINIT\_FOCK}} \label{MINIT_FOCK}
\vspace*{-12pt}
\begin{columns}
\column{0.4\linewidth}
\begin{block}{Type}
Integer
\end{block}

\begin{block}{Default}
2
\end{block}

\column{0.4\linewidth}
\begin{block}{Unit}
No unit
\end{block}

\begin{block}{Example}
\texttt{MINIT\_FOCK}: 3
\end{block}
\end{columns}

\begin{block}{Description}
The minimum number of iterations for Hartree-Fock outer loop.
\end{block}

\begin{block}{Remark}
Only active when using hybrid functionals, like PBE0 and HSE. 
\end{block}

\end{frame}
%%%%%%%%%%%%%%%%%%%%%%%%%%%%%%%%%%%%%%%%%%%%%%%%%%%%%%%%%%%%%%%%%%%%%%%%%%%%%%%%%%%%%%%%%%%%%


%%%%%%%%%%%%%%%%%%%%%%%%%%%%%%%%%%%%%%%%%%%%%%%%%%%%%%%%%%%%%%%%%%%%%%%%%%%%%%%%%%%%%%%%%%%%%
\begin{frame}[allowframebreaks]{\texttt{TOL\_SCF\_INIT}} \label{TOL_SCF_INIT}
\vspace*{-12pt}
\begin{columns}
\column{0.4\linewidth}
\begin{block}{Type}
Double
\end{block}

\begin{block}{Default}
$\max($\hyperlink{TOL_FOCK}{\texttt{TOL\_FOCK}}$\times 10,0.001)$
\end{block}

\column{0.4\linewidth}
\begin{block}{Unit}
No unit
\end{block}

\begin{block}{Example}
\texttt{TOL\_SCF\_INIT}: 1e-6
\end{block}
\end{columns}

\begin{block}{Description}
The initial SCF tolerance for PBE iteration when using hybrid functionals. 
\end{block}

\begin{block}{Remark}
Only active when using hybrid functionals, like PBE0 and HSE. Change the \texttt{TOL\_SCF\_INIT} to change the initial guess for Hartree Fock outer loop.
\end{block}

\end{frame}
%%%%%%%%%%%%%%%%%%%%%%%%%%%%%%%%%%%%%%%%%%%%%%%%%%%%%%%%%%%%%%%%%%%%%%%%%%%%%%%%%%%%%%%%%%%%%

%%%%%%%%%%%%%%%%%%%%%%%%%%%%%%%%%%%%%%%%%%%%%%%%%%%%%%%%%%%%%%%%%%%%%%%%%%%%%%%%%%%%%%%%%%%%%
\begin{frame}[allowframebreaks]{\texttt{ACE\_FLAG}} \label{ACE_FLAG}
\vspace*{-12pt}
\begin{columns}
\column{0.4\linewidth}
\begin{block}{Type}
Integer
\end{block}

\begin{block}{Default}
1
\end{block}

\column{0.4\linewidth}
\begin{block}{Unit}
No unit
\end{block}

\begin{block}{Example}
\texttt{ACE\_FLAG}: 0
\end{block}
\end{columns}

\begin{block}{Description}
Use ACE operator to accelarte the hybrid calculation.
\end{block}

\begin{block}{Remark}
Without ACE operator, the hybrid calculation will be way slower than with it on depending on the system size.
\end{block}

\end{frame}
%%%%%%%%%%%%%%%%%%%%%%%%%%%%%%%%%%%%%%%%%%%%%%%%%%%%%%%%%%%%%%%%%%%%%%%%%%%%%%%%%%%%%%%%%%%%%


%%%%%%%%%%%%%%%%%%%%%%%%%%%%%%%%%%%%%%%%%%%%%%%%%%%%%%%%%%%%%%%%%%%%%%%%%%%%%%%%%%%%%%%%%%%%%
\begin{frame}[allowframebreaks]{\texttt{EXX\_METHOD}} \label{EXX_METHOD}
\vspace*{-12pt}
\begin{columns}
\column{0.4\linewidth}
\begin{block}{Type}
String
\end{block}

\begin{block}{Default}
\texttt{FOURIER\_SPACE}
\end{block}

\column{0.4\linewidth}
\begin{block}{Unit}
No unit
\end{block}

\begin{block}{Example}
\texttt{EXX\_METHOD}: \texttt{REAL\_SPACE}
\end{block}
\end{columns}

\begin{block}{Description}
Methods to solve Poisson's equation in Exact Exchange part. Options include using FFT to solve it in Fourier space and using linear solver, like CG, to solve in Real space.
\end{block}

\begin{block}{Remark}
Only active when using hybrid functionals for molecule simulation, like PBE0 and HSE. 
\texttt{FOURIER\_SPACE} method is much faster than \texttt{REAL\_SPACE} method.
\end{block}

\end{frame}
%%%%%%%%%%%%%%%%%%%%%%%%%%%%%%%%%%%%%%%%%%%%%%%%%%%%%%%%%%%%%%%%%%%%%%%%%%%%%%%%%%%%%%%%%%%%%

%%%%%%%%%%%%%%%%%%%%%%%%%%%%%%%%%%%%%%%%%%%%%%%%%%%%%%%%%%%%%%%%%%%%%%%%%%%%%%%%%%%%%%%%%%%%%
\begin{frame}[allowframebreaks]{\texttt{EXX\_MEM}} \label{EXX_MEM}
\vspace*{-12pt}
\begin{columns}
\column{0.4\linewidth}
\begin{block}{Type}
Integer
\end{block}

\begin{block}{Default}
20
\end{block}

\column{0.4\linewidth}
\begin{block}{Unit}
No unit
\end{block}

\begin{block}{Example}
\texttt{EXX\_MEM}: 0
\end{block}
\end{columns}

\begin{block}{Description}
Number of Poisson's equations to be solved in each process at a time when creating exact exchange operator or ACE operator. Typically, when \texttt{EXX\_MEM} is larger than 20, the speed of code is barely affected. When it is 0, all Poisson's equations are solved together and it hits the fastest speed but largest memory requirement. 
\end{block}

\end{frame}
%%%%%%%%%%%%%%%%%%%%%%%%%%%%%%%%%%%%%%%%%%%%%%%%%%%%%%%%%%%%%%%%%%%%%%%%%%%%%%%%%%%%%%%%%%%%%


%%%%%%%%%%%%%%%%%%%%%%%%%%%%%%%%%%%%%%%%%%%%%%%%%%%%%%%%%%%%%%%%%%%%%%%%%%%%%%%%%%%%%%%%%%%%%
\begin{frame}[allowframebreaks]{\texttt{EXX\_FRAC}} \label{EXX_FRAC}
\vspace*{-12pt}
\begin{columns}
\column{0.4\linewidth}
\begin{block}{Type}
Double
\end{block}

\begin{block}{Default}
0.25 for PBE0 and HSE
\end{block}

\column{0.4\linewidth}
\begin{block}{Unit}
No unit
\end{block}

\begin{block}{Example}
\texttt{EXX\_FRAC}: 0.3
\end{block}
\end{columns}

\begin{block}{Description}
Fraction of exact exchange in hybrid functional, e.g. PBE0 and HSE, while the fraction of PBE is 1-\texttt{EXX\_FRAC}
\end{block}

\end{frame}
%%%%%%%%%%%%%%%%%%%%%%%%%%%%%%%%%%%%%%%%%%%%%%%%%%%%%%%%%%%%%%%%%%%%%%%%%%%%%%%%%%%%%%%%%%%%%


%%%%%%%%%%%%%%%%%%%%%%%%%%%%%%%%%%%%%%%%%%%%%%%%%%%%%%%%%%%%%%%%%%%%%%%%%%%%%%%%%%%%%%%%%%%%%
\begin{frame}[allowframebreaks]{\texttt{EXX\_ACE\_VALENCE\_STATES}} \label{EXX_ACE_VALENCE_STATES}
\vspace*{-12pt}
\begin{columns}
\column{0.4\linewidth}
\begin{block}{Type}
Integer
\end{block}

\begin{block}{Default}
3
\end{block}

\column{0.4\linewidth}
\begin{block}{Unit}
No unit
\end{block}

\begin{block}{Example}
\texttt{EXX\_ACE\_VALENCE\_STATES}: 1
\end{block}
\end{columns}

\begin{block}{Description}
Control of number of unoccupied states used to construct ACE operator.
\end{block}

\begin{block}{Remark}
Only active when using hybrid functionals with ACE operator. 
\end{block}

\end{frame}
%%%%%%%%%%%%%%%%%%%%%%%%%%%%%%%%%%%%%%%%%%%%%%%%%%%%%%%%%%%%%%%%%%%%%%%%%%%%%%%%%%%%%%%%%%%%%


%%%%%%%%%%%%%%%%%%%%%%%%%%%%%%%%%%%%%%%%%%%%%%%%%%%%%%%%%%%%%%%%%%%%%%%%%%%%%%%%%%%%%%%%%%%%%
\begin{frame}[allowframebreaks]{\texttt{EXX\_DOWNSAMPLING}} \label{EXX_DOWNSAMPLING}
\vspace*{-12pt}
\begin{columns}
\column{0.4\linewidth}
\begin{block}{Type}
Integer
\end{block}

\begin{block}{Default}
1 1 1
\end{block}

\column{0.4\linewidth}
\begin{block}{Unit}
No unit
\end{block}

\begin{block}{Example}
\texttt{EXX\_DOWNSAMPLING}: 1 2 3
\end{block}
\end{columns}

\begin{block}{Description}
Down-sampling of k-points grids. There should be 3 nonnegative integers. 0 means using 0 k-point in that direction, 
requiring 0 is one of the k-point after time-reversal symmetry in that direction. 
Positive value should be a factor of the number of grid points in that direction. 
\end{block}

\end{frame}
%%%%%%%%%%%%%%%%%%%%%%%%%%%%%%%%%%%%%%%%%%%%%%%%%%%%%%%%%%%%%%%%%%%%%%%%%%%%%%%%%%%%%%%%%%%%%


%%%%%%%%%%%%%%%%%%%%%%%%%%%%%%%%%%%%%%%%%%%%%%%%%%%%%%%%%%%%%%%%%%%%%%%%%%%%%%%%%%%%%%%%%%%%%
\begin{frame}[allowframebreaks]{\texttt{EXX\_DIVERGENCE}} \label{EXX_DIVERGENCE}
\vspace*{-12pt}
\begin{columns}
\column{0.35\linewidth}
\begin{block}{Type}
String
\end{block}

\begin{block}{Default}
SPHERICAL
\end{block}

\column{0.55\linewidth}
\begin{block}{Unit}
No unit
\end{block}

\begin{block}{Example}
\texttt{EXX\_DIVERGENCE}: \texttt{AUXILIARY}
\end{block}
\end{columns}

\begin{block}{Description}
Treatment of divergence in exact exchange. Options are \texttt{SPHERICAL} (spherical truncation), 
\texttt{AUXILIARY} (auxiliary function method) and \texttt{ERFC} (erfc screening).
\end{block}

\begin{block}{Remark}
For systems with cube-like geometry, both methods converge fast. For slab and wire, auxiliary function method is a better option. 
ERFC screening is the default option for HSE in bulk and molecule simulation.
\end{block}

\end{frame}
%%%%%%%%%%%%%%%%%%%%%%%%%%%%%%%%%%%%%%%%%%%%%%%%%%%%%%%%%%%%%%%%%%%%%%%%%%%%%%%%%%%%%%%%%%%%%